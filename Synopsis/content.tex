\pdfbookmark{Общая характеристика работы}{characteristic}             % Закладка pdf
\section*{Общая характеристика работы}

\newcommand{\actuality}{\pdfbookmark[1]{Актуальность}{actuality}\underline{\textbf{\actualityTXT}}}
\newcommand{\progress}{\pdfbookmark[1]{Разработанность темы}{progress}\underline{\textbf{\progressTXT}}}
\newcommand{\aim}{\pdfbookmark[1]{Цели}{aim}\underline{{\textbf\aimTXT}}}
\newcommand{\tasks}{\pdfbookmark[1]{Задачи}{tasks}\underline{\textbf{\tasksTXT}}}
\newcommand{\aimtasks}{\pdfbookmark[1]{Цели и задачи}{aimtasks}\aimtasksTXT}
\newcommand{\novelty}{\pdfbookmark[1]{Научная новизна}{novelty}\underline{\textbf{\noveltyTXT}}}
\newcommand{\influence}{\pdfbookmark[1]{Практическая значимость}{influence}\underline{\textbf{\influenceTXT}}}
\newcommand{\methods}{\pdfbookmark[1]{Методология и методы исследования}{methods}\underline{\textbf{\methodsTXT}}}
\newcommand{\defpositions}{\pdfbookmark[1]{Положения, выносимые на защиту}{defpositions}\underline{\textbf{\defpositionsTXT}}}
\newcommand{\reliability}{\pdfbookmark[1]{Достоверность}{reliability}\underline{\textbf{\reliabilityTXT}}}
\newcommand{\probation}{\pdfbookmark[1]{Апробация}{probation}\underline{\textbf{\probationTXT}}}
\newcommand{\contribution}{\pdfbookmark[1]{Личный вклад}{contribution}\underline{\textbf{\contributionTXT}}}
\newcommand{\publications}{\pdfbookmark[1]{Публикации}{publications}\underline{\textbf{\publicationsTXT}}}



{\actuality}

Активные ядра галактик ...

% \ifsynopsis
% Этот абзац появляется только в~автореферате.
% Для формирования блоков, которые будут обрабатываться только в~автореферате,
% заведена проверка условия \verb!\!\verb!ifsynopsis!.
% Значение условия задаётся в~основном файле документа (\verb!synopsis.tex! для
% автореферата).
% \else
% Этот абзац появляется только в~диссертации.
% Через проверку условия \verb!\!\verb!ifsynopsis!, задаваемого в~основном файле
% документа (\verb!dissertation.tex! для диссертации), можно сделать новую
% команду, обеспечивающую появление цитаты в~диссертации, но~не~в~автореферате.
% \fi

% {\progress} 
% Этот раздел должен быть отдельным структурным элементом по
% ГОСТ, но он, как правило, включается в описание актуальности
% темы. Нужен он отдельным структурынм элемементом или нет ---
% смотрите другие диссертации вашего совета, скорее всего не нужен.

{\aim}

Целью данной работы является \ldots

{\novelty}
% \begin{enumerate}
%   \item Впервые \ldots
%   \item Впервые \ldots
%   \item Было выполнено оригинальное исследование \ldots
% \end{enumerate}

% {\influence} \ldots

% {\methods} \ldots

{\defpositions}
\begin{enumerate}
  \item
  \item
  \item
\end{enumerate}
% В папке Documents можно ознакомиться в решением совета из Томского ГУ
% в~файле \verb+Def_positions.pdf+, где обоснованно даются рекомендации
% по~формулировкам защищаемых положений.

{\reliability}

{\probation}


%\publications\ Основные результаты по теме диссертации изложены в ХХ печатных изданиях~\cite{Sokolov,Gaidaenko,Lermontov,Management},
%Х из которых изданы в журналах, рекомендованных ВАК~\cite{Sokolov,Gaidaenko}, 
%ХХ --- в тезисах докладов~\cite{Lermontov,Management}.

% \ifnumequal{\value{bibliosel}}{0}{% Встроенная реализация с загрузкой файла через движок bibtex8
%     \publications\
%
% %     Основные результаты по теме диссертации изложены в XX печатных изданиях,
% %     X из которых изданы в журналах, рекомендованных ВАК,
% %     X "--- в тезисах докладов.%
% }{% Реализация пакетом biblatex через движок biber
% %Сделана отдельная секция, чтобы не отображались в списке цитированных материалов
%     \begin{refsection}% Подсчет и нумерация авторских работ. Засчитываются только те, которые были прописаны внутри \nocite{}.
%         %Чтобы сменить порядок разделов в сгрупированном списке литературы необходимо перетасовать следующие три строчки, а также команды в разделе \newcommand*{\insertbiblioauthorgrouped} в файле biblio/biblatex.tex
%         \printbibliography[heading=countauthorvak, env=countauthorvak, keyword=biblioauthorvak, section=1]%
% %         \printbibliography[heading=countauthorconf, env=countauthorconf, keyword=biblioauthorconf, section=1]%
% %         \printbibliography[heading=countauthornotvak, env=countauthornotvak, keyword=biblioauthornotvak, section=1]%
% %         \printbibliography[heading=countauthor, env=countauthor, keyword=biblioauthor, section=1]%
%         \nocite{Voitsik_2018, Kutkin_2018}
%
%
%     \end{refsection}
% %     \begin{refsection}[vak,papers,conf]%Блок, позволяющий отобрать из всех работ автора наиболее значимые, и только их вывести в автореферате, но считать в блоке выше общее число работ
% %         \printbibliography[heading=countauthorvak, env=countauthorvak, keyword=biblioauthorvak, section=2]%
% %         \printbibliography[heading=countauthornotvak, env=countauthornotvak, keyword=biblioauthornotvak, section=2]%
% %         \printbibliography[heading=countauthorconf, env=countauthorconf, keyword=biblioauthorconf, section=2]%
% %         \printbibliography[heading=countauthor, env=countauthor, keyword=biblioauthor, section=2]%
% %         \nocite{vakbib2}%vak
% %         \nocite{bib1}%notvak
% %         \nocite{confbib1}%conf
% %     \end{refsection}
% }
% При использовании пакета \verb!biblatex! для автоматического подсчёта
% количества публикаций автора по теме диссертации, необходимо
% их~здесь перечислить с использованием команды \verb!\nocite!.


{\contribution}
Основные результаты по теме диссертации изложены в~\arabic{citeauthorvak}~научных статьях \cite{Voitsik_2018, Kardashev_2013, Kutkin_2018}, в рецензируемых журналах и изданиях, рекомендованных ВАК.

% \section*{Объем и структура диссертации}
%
% Диссертация состоит из введения, трёх глав, заключения и списка литературы.
% Полный объём диссертации составляет
% \formbytotal{TotPages}{страниц}{у}{ы}{}, включая
% \formbytotal{totalcount@figure}{рисун}{ок}{ка}{ков} и
% \formbytotal{totalcount@table}{таблиц}{у}{ы}{}.   Список литературы содержит
% \formbytotal{citenum}{наименован}{ие}{ия}{ий}.
%
 % Характеристика работы по структуре во введении и в автореферате
                              % не отличается (ГОСТ Р 7.0.11, пункты 5.3.1 и 9.2.1), потому её
                              % загружаем из одного и того же внешнего файла, предварительно задав
                              % форму выделения некоторым параметрам

\pdfbookmark{Содержание работы}{description}                          % Закладка pdf
\section*{Содержание работы}

Во \textbf{введении} приводится общее содержание работы, обосновывается актуальность
исследований, формулируется цель, ставятся задачи работы, излагается научная новизна
и практическая значимость представляемой работы.

\textbf{Первая глава} посвящена исследованию эффекта видимого сдвига положения ядра с частотой в
ультракомпактных квазарах.

\textbf{Вторая глава} посвящена \dots

\textbf{Третья глава} посвящена \dots
Ссылка: \cite{Blandford_Konigl_1979}

\pdfbookmark{Заключение}{conclusion}                                  % Закладка pdf
В \underline{\textbf{заключении}} сформулированы основные результаты работы.
