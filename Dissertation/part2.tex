\chapter{Наземно-космический интерферометр ``РадиоАстрон''} \label{chapt2}

Статья: \cite{Kardashev_2013_rus}

Автоматический космический аппарат ``Спектр-Р'' предназначен для установки на нем космического радиотелескопа, используемого в качестве орбитального плеча наземно-космического
радиоинтерферометра со сверхдлинными базами (РСДБ). В его состав входят базовый служебный модуль
``Навигатор'' (головная организация~--- НПО им. С.А. Лавочкина) [21] и научный комплекс, состоящий
из научной аппаратуры космического радиотелескопа международного проекта ``РадиоАстрон'' (головная
организация~--- АКЦ ФИАН) и параболической антенны диаметром 10 м (совместная разработка НПО им.
С.А. Лавочкина и АКЦ ФИАН) [22, 23]. Помимо этого, здесь же размещена научная аппаратура проекта
``Плазма-Ф'' (головная организация~--- Институт космических исследований РАН), предназначенная для
исследований космической плазмы в пределах орбиты аппарата ``Спектр-Р'' (эта аппаратура и
эксперимент описаны в публикациях [26, 27]).


\section{Измерение чувствительности КРТ по астрономическим источникам}

Приемная система радиотелескопа состоит из 8
приемников на 4 диапазона: 1.35, 6.2, 18 и 92 см,
по два приемника на диапазон --- для левой и правой круговой поляризации излучения. На входы
каждой такой пары приемников поступают сигналы
от формирователей круговых поляризаций соответствующего диапазона, которые конструктивно
объединены с облучателями антенны в общий четырехдиапазонный соосный блок антенных облучателей
(БАО).


\section{Первые лепестки}

\section{Оценки точности восстановления орбиты КРТ по интерферометрическим наблюдениям}
