\chapter{Наземно-космический интерферометр ``РадиоАстрон''} \label{chapt2}


Автоматический космический аппарат ``Спектр-Р'' предназначен для установки на нем космического
радиотелескопа, используемого в качестве орбитального плеча наземно-космического
радиоинтерферометра со сверхдлинными базами (РСДБ). В его состав входят базовый служебный модуль
``Навигатор'' (головная организация~--- НПО им. С.А. Лавочкина) [21] и научный комплекс, состоящий
из научной аппаратуры космического радиотелескопа международного проекта ``РадиоАстрон'' (головная
организация~--- АКЦ ФИАН) и параболической антенны диаметром 10 м (совместная разработка НПО им.
С.А. Лавочкина и АКЦ ФИАН) [22, 23]. Помимо этого, здесь же размещена научная аппаратура проекта
``Плазма-Ф'' (головная организация~--- Институт космических исследований РАН), предназначенная для
исследований космической плазмы в пределах орбиты аппарата ``Спектр-Р'' (эта аппаратура и
эксперимент описаны в публикациях [26, 27]).


\section{Измерение чувствительности КРТ по астрономическим источникам}

Статьи: \cite{Kardashev_2013_rus, Kovalev_2014_rus}

Приемная система радиотелескопа состоит из 8 приемников на 4 диапазона: 1.35, 6.2, 18 и
\SI{92}{\cm}, по два приемника на диапазон~--- для левой и правой круговой поляризации излучения. На
входы каждой такой пары приемников поступают сигналы от формирователей круговых поляризаций
соответствующего диапазона, которые конструктивно объединены с облучателями антенны в общий
четырехдиапазонный соосный блок антенных облучателей (БАО).

Облучатель в диапазоне \SI{1.35}{\cm} представляет собой круглый волновод, переходящий в волноводный
формирователь-разделитель круговых поляризаций с двумя прямоугольными волноводами на выходе.
Облучатели в остальных диапазонах~--- кольцевые щелевые, радиус колец которых увеличивается с ростом
длины волны. Кольцевые щели облучателей соосны друг с другом и с круглым волноводом.
Формирователи-разделители поляризаций в диапазонах 6.2, 18 и \SI{92}{\cm} --- полосковые с коаксиальными
выходами. Восемь выходов формирователей поляризаций БАО соединены с блоками входных малошумящих
усилителей (МШУ) приемников соответствующих диапазонов отрезками линий~--- волноводных в диапазоне
1.35\,см и коаксиальных в остальных диапазонах. Для повышения чувствительности БАО и МШУ всех
диапазонов, кроме диапазона \SI{92}{\cm}, охлаждаются радиационным способом до температур около
\SI{150}{\kelvin} (БАО) и \SI{130}{\kelvin} (МШУ). Для этого МШУ диапазонов 1.35, 6.2 и 18\,см
вынесены из приемников, размещенных в герметичном фокальном контейнере, и установлены на отдельной
``холодной плите'' в открытом космосе в тени конструкции КРТ. Неохлаждаемый МШУ диапазона 92\,см
находится внутри термостатируемого приемника в фокальном контейнере при температуре около
\SI{30}{\degreeCelsius}. Калибровочные сигналы от внутренних генераторов шума (ГШ) поступают на
входы МШУ из приемников по отдельным коаксиальным линиям.

На выходе каждого приемника, после усиления
и гетеродинного преобразования сигналов с входных полос частот в те же полосы на промежуточной
частоте (ПЧ) вблизи 512\,МГц, формируются два
вида сигналов: высокочастотный интерферометрический на ПЧ и низкочастотный радиометрический
сигнал. Последний образуется в радиометрическом тракте приемника из сигнала на промежуточной частоте
после его квадратичного детектирования, усиления и усреднения на интервалах времени около секунды.
Радиометрический сигнал дает
возможность быстрого и эффективного контроля
функционирования КРТ и проведения антенных
измерений в режиме одиночного телескопа.

Высокочастотный сигнал на ПЧ с выхода приемника поступает на селектор ПЧ, в котором из
8 ПЧ-выходов от всех приемников выбираются
два для их последующего гетеродинного преобразования к более низким частотам и формирования
непрерывного цифрового потока интерферометрических видеоданных для передачи на Землю. Фазовая
стабильность всех преобразований обеспечивается бортовым водородным стандартом частоты
или, в альтернативном штатном режиме, замкнутой фазовой петлей связи с наземным водородным
стандартом частоты. Низкочастотные радиометрические сигналы с выходов всех приемников сразу
поступают в бортовую телеметрическую систему
(ТМС) космического аппарата. ТМС собирает радиометрические и другие низкочастотные данные
со всей научной и служебной аппаратуры и формирует другой непрерывный поток данных с КРТ~--- поток
телеметрической информации.

Телеметрические данные передаются на Землю
через телеметрический канал (в режимах реального
времени или с разделением времени, если данные
помещаются в бортовое запоминающее устройство
для временного хранения), который использует
малонаправленные антенны на борту и штатные
измерительные пункты на Земле. Поток данных
в режиме интерферометра передается на Землю
в реальном времени через специальный высокоскоростой радиоканал передачи научных данных
на частоте 15\,ГГц (канал ВИРК). Передающая
1.5-м параболическая антенна ВИРК размещена
с тыльной стороны КРТ и КА на днище КА и, в
ограниченном интервале углов, может наводиться
на приемную 22-м параболическую антенну наземной станции слежения в Пущинской радиоастрономической
обсерватории Астрокосмического центра ФИАН или на другую станцию слежения.
Эти антенны используются также как приемопередающие антенны на частотах
\num{8.4}/\SI{7.2}{\giga\hertz} для работы в режиме замкнутой фазовой петли связи.



\section{Первые лепестки}

Статьи: \cite{Kardashev_2013_rus}



\section{Оценки точности восстановления орбиты КРТ по интерферометрическим наблюдениям}

Статьи: \cite{Lobanov_2015,Zakhvatkin_2018}

\section{Картографирование квазара 0716+714}

Статьи: \cite{Kardashev_2013_rus}

Для активной галактики 0716+714 --- одного из
самых быстропеременных внегалактических объектов --- успешное детектирование интерференционных
лепестков в том же диапазоне было выполнено для многих проекций баз, от примерно 1.5 до
более 5 диаметров Земли. Международной рабочей группе по Ранней научной программе проекта
“РадиоАстрон” удалось по этим данным восстановить изображение объекта (рис. ??) и измерить
параметры видимого ядра. Ширина у основания
струи в ядре объекта оказалась равной примерно
70 мксек. дуги или 0.3\,пк, а яркостная температура --- $2 \times 10^{12}\,\text{K}$. Заметим, что
эти параметры измерены в момент минимума активности этого объекта.
