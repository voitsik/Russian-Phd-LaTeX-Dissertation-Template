\chapter{Наземно-космический интерферометр ``РадиоАстрон''} \label{chapt2}

% TODO: ужасный язык -- переписать
Автоматический космический аппарат ``Спектр-Р'' предназначен для установки на нем космического
радиотелескопа, используемого в качестве орбитального плеча наземно-космического
радиоинтерферометра со сверхдлинными базами (РСДБ). В его состав входят базовый служебный модуль
``Навигатор'' (головная организация~--- НПО им. С.А. Лавочкина) [21] и научный комплекс, состоящий
из научной аппаратуры космического радиотелескопа международного проекта ``РадиоАстрон'' (головная
организация~--- АКЦ ФИАН) и параболической антенны диаметром 10 м (совместная разработка НПО им.
С.А. Лавочкина и АКЦ ФИАН) [22, 23]. Помимо этого, здесь же размещена научная аппаратура проекта
``Плазма-Ф'' (головная организация~--- Институт космических исследований РАН), предназначенная для
исследований космической плазмы в пределах орбиты аппарата ``Спектр-Р'' (эта аппаратура и
эксперимент описаны в публикациях [26, 27]).



\section{Измерение чувствительности КРТ по астрономическим источникам}

Статьи: \cite{Kardashev_2013_rus, Kovalev_2014_rus}

\subsection{Приемная система КРТ}

% Kardashev_2013_rus
Приемная система радиотелескопа состоит из 8 приемников на 4 диапазона: 1.35, 6.2, 18 и
\SI{92}{\cm}, по два приемника на диапазон~--- для левой и правой круговой поляризации излучения. На
входы каждой такой пары приемников поступают сигналы от формирователей круговых поляризаций
соответствующего диапазона, которые конструктивно объединены с облучателями антенны в общий
четырехдиапазонный соосный блок антенных облучателей (БАО).

% Kovalev_2014_rus
Бортовой научный комплекс включает в себя 4 радиоастрономических супергетеродинных приемника~--- на
диапазоны 92, 18, 6.2 и \SI{1.35}{\cm}. Приемник диапазона \SI{1.35}{\cm} обеспечивает также прием
сигнала в 8 переключаемых поддиапазонах от 1.7 до 1.2 см с помощью выбора одного из поддиапазонов
соответствующими командами. Блоки входных малошумящих усилителей (МШУ) приемников всех диапазонов,
кроме \SI{92}{\cm}, вынесены в открытый космос и размещены на ``холодной плите'', охлаждаемой до
температуры \SI{130}{\kelvin} радиационным способом. Каждый приемник состоит из 2-х идентичных
каналов, на входы которых от антенны через блок антенных облучателей (БАО) с разделителями
поляризаций поступает излучение в левой и правой круговых поляризациях. Каждый канал имеет два
параллельных выхода: 1) радиометрический выход~--- с продетектированным сигналом, который поступает
на телеметрическую систему космического аппарата и используется в антенных измерениях, 2)
интерферометрический выход~--- с сигналом на промежуточной частоте, который после дальнейших
преобразований используется в работе наземно-космического интерферометра.

% Kardashev_2013_rus (5.1)
Облучатель в диапазоне \SI{1.35}{\cm} представляет собой круглый волновод, переходящий в волноводный
формирователь-разделитель круговых поляризаций с двумя прямоугольными волноводами на выходе.
Облучатели в остальных диапазонах~--- кольцевые щелевые, радиус колец которых увеличивается с ростом
длины волны. Кольцевые щели облучателей соосны друг с другом и с круглым волноводом.
Формирователи-разделители поляризаций в диапазонах 6.2, 18 и \SI{92}{\cm} --- полосковые с
коаксиальными выходами. Восемь выходов формирователей поляризаций БАО соединены с блоками входных
малошумящих усилителей (МШУ) приемников соответствующих диапазонов отрезками линий~--- волноводных в
диапазоне \SI{1.35}{\cm} и коаксиальных в остальных диапазонах. Для повышения чувствительности БАО и
МШУ всех диапазонов, кроме диапазона \SI{92}{\cm}, охлаждаются радиационным способом до температур
около \SI{150}{\kelvin} (БАО) и \SI{130}{\kelvin} (МШУ). Для этого МШУ диапазонов 1.35, 6.2 и
\SI{18}{\cm} вынесены из приемников, размещенных в герметичном фокальном контейнере, и установлены
на отдельной ``холодной плите'' в открытом космосе в тени конструкции КРТ. Неохлаждаемый МШУ
диапазона \SI{92}{\cm} находится внутри термостатируемого приемника в фокальном контейнере при
температуре около \SI{30}{\degreeCelsius}. Калибровочные сигналы от внутренних генераторов шума (ГШ)
поступают на входы МШУ из приемников по отдельным коаксиальным линиям.

На выходе каждого приемника, после усиления и гетеродинного преобразования сигналов с входных полос
частот в те же полосы на промежуточной частоте (ПЧ) вблизи \SI{512}{\MHz}, формируются два вида
сигналов: высокочастотный интерферометрический на ПЧ и низкочастотный радиометрический сигнал.
Последний образуется в радиометрическом тракте приемника из сигнала на промежуточной частоте после
его квадратичного детектирования, усиления и усреднения на интервалах времени около секунды.
Радиометрический сигнал дает возможность быстрого и эффективного контроля функционирования КРТ и
проведения антенных измерений в режиме одиночного телескопа.

Высокочастотный сигнал на ПЧ с выхода приемника поступает на селектор ПЧ, в котором из 8 ПЧ-выходов
от всех приемников выбираются два для их последующего гетеродинного преобразования к более низким
частотам и формирования непрерывного цифрового потока интерферометрических видеоданных для передачи
на Землю. Фазовая стабильность всех преобразований обеспечивается бортовым водородным стандартом
частоты или, в альтернативном штатном режиме, замкнутой фазовой петлей связи с наземным водородным
стандартом частоты. Низкочастотные радиометрические сигналы с выходов всех приемников сразу
поступают в бортовую телеметрическую систему (ТМС) космического аппарата. ТМС собирает
радиометрические и другие низкочастотные данные со всей научной и служебной аппаратуры и формирует
другой непрерывный поток данных с КРТ~--- поток телеметрической информации.

Телеметрические данные передаются на Землю через телеметрический канал (в режимах реального времени
или с разделением времени, если данные помещаются в бортовое запоминающее устройство для временного
хранения), который использует малонаправленные антенны на борту и штатные измерительные пункты на
Земле. Поток данных в режиме интерферометра передается на Землю в реальном времени через специальный
высокоскоростой радиоканал передачи научных данных на частоте \SI{15}{\GHz} (канал ВИРК). Передающая
1.5-м параболическая антенна ВИРК размещена с тыльной стороны КРТ на днище космического аппарата и,
в ограниченном интервале углов, может наводиться на приемную 22-м параболическую антенну наземной
станции слежения в Пущинской радиоастрономической обсерватории Астрокосмического центра ФИАН или на
другую станцию слежения. Эти антенны используются также как приемопередающие антенны на частотах
\num{8.4}/\SI{7.2}{\GHz} для работы в режиме замкнутой фазовой петли связи.

\subsection{Чувствительность радиотелескопа}

Чувствительность по антенной температуре $\sigma_T$
и спектральной плотности потока $\sigma_F$ КРТ как одиночного телескопа с приемником
супергетеродинного типа в радиометрическом режиме, который в основном и используется в антенных
измерениях, определяется известными соотношениями через
эквивалентную температуру шумов системы $T_{sys}$ и
эффективную площадь $A_{eff}$ радиотелескопа [??]:

\begin{equation}
 \sigma_T = T_{sys} \sqrt{\frac{2}{\Delta\nu \tau} + \left(\frac{\sigma_G}{G}\right)^2}\,,
\end{equation}

\begin{equation}
 \sigma_F = \frac{2 k_B \sigma_T}{A_{eff}}
\end{equation}

Здесь $\sigma_G/G$~--- относительная нестабильность коэффициента усиления $G$ приемного тракта,
$\Delta \nu$~--- ширина полосы частот по ПЧ (в нашем случае
равная ширине полосы входных частот, но вблизи
промежуточной частоты 512 МГц) и $\tau$~--- время интегрирования сигнала после квадратичного
детектирования (все эти величины относятся к радиометрическому тракту), $k_B$~--- постоянная
Больцмана.

Чувствительность $\sigma_{VLBI}$ двухантенного интерферометра
удобно выразить через $T_{sys}$ и $A_{eff}$, которые определяются для каждого телескопа
\cite{VLBIbook}:

\begin{equation}
 \sigma_{VLBI} = b \sqrt{\frac{F_{sys,1}F_{sys,2}}{2 \Delta \nu_{IF} \Delta t_c}}\,,
\end{equation}

\begin{equation}
 F_{sys,i} = \frac{2 k_B T_{sys,i}}{A_{eff,i}} \,.
\end{equation}

Здесь коэффициент $b$ зависит от количества уровней квантования при оцифровке сигнала, $\Delta
\nu_{IF}$~--- ширина полосы регистрируемого сигнала, $\Delta t_c$~--- время усреднения данных при
посткорреляционном анализе, а произведение $2 \Delta \nu_{IF} \Delta t_c$ равно количеству
независимых отсчетов за время усреднения. Величина $F_{sys}$~--- это эквивалентная спектральная
плотность потока системы или, как ее часто обозначают, SEFD~--- ``System Equivalence Flux Density''.
Она численно равна спектральной плотности потока такого точечного источника, который дает
на выходе приемника мощность шума в два раза больше, чем в случае отсутствия источника. Удобство
использования величины $F_{sys}$ связано с простотой ее прямого измерения с помощью квазиточечного
источника c известной спектральной плотностью потока излучения $F_s$:

\begin{equation}
 F_{sys}  = F_s \frac{U_{sys}}{U_s g} \,,
\end{equation}
где $U_{sys}/U_s = T_{sys}/T_s$~--- непосредственно измеряемое на радиометрическом выходе
приемника отношение продетектированных откликов на шумы
системы $T_{sys}$ и на прохождение источника c антенной температурой $T_s$ через диаграмму
направленности радиотелескопа. $g \geqslant 1$~--- коэффициент частичного разрешения источника,
который может быть рассчитан численно при известной диаграмме направленности радиотелескопа
и известном распределении яркостной температуры по объекту, для точечного источника $g = 1$.

\subsection{Измерения}

Цель автономных антенных измерений состоит
в получении основных параметров космического радиотелескопа в полете. Эти параметры используются в
работе наземно-космического радиоинтерферометра, а также для контроля состояния КРТ.
В данной работе решались следующие задачи:

\begin{itemize}
 \item Измерение шумовых характеристик радиотелескопа в рабочих диапазонах длин волн 92, 18,
6.2 и \SI{1.35}{\cm}: эквивалентную шумовую температуру системы $T_{sys}$ и эквивалентную плотность
потока излучения системы $F_{sys}$.
 \item Измерение эффективной площади $A_{eff}$ на
рабочих длинах волн с помощью наблюдений астрономических калибровочных источников радиоизлучения в
непрерывном спектре, используя при этом значения шумовой температуры калибровочного сигнала от
внутреннего ГШ, определенные по результатам наземных испытаний.
\end{itemize}

Представленные ниже радиоастрономические
измерения основных параметров орбитального телескопа (``антенные измерения'') были частью программы
летных испытаний космического аппарата
и КРТ, выполнявшейся в первые полгода после
запуска. К началу проведения антенных
измерений в середине сентября 2011 г. были
выполнены следующие технические операции.

\begin{itemize}
 \item Проверена работоспособность антенны и
приемников в радиометрическом режиме для каждого из двух поляризационных каналов приемников
(левой и правой круговых поляризаций) в рабочих диапазонах длин волн 92, 18, 6.2 и \SI{1.35}{\cm}.
 \item Проверена реализуемость управления и проектных режимов движения КРТ в инерциальной
прямоугольной системе координат XYZ, жестко связанной с центром масс космического аппарата,
по заданиям с Земли и с бортового запоминающего устройства.
 \item  Проверена возможность представления заданий и получения результатов в астрономической
экваториальной системе координат прямое восхождение--склонение на эпоху J2000.
\end{itemize}

Ось X космического аппарата выбрана совпадающей с
геометрической осью зеркала КРТ, ось Y~--- параллельно оси поворота солнечных батарей. Режимы
движения КРТ аналогичны режимам наземных радиотелескопов: ``Наведение'', ``Сопровождение'',
``Сканирование''. Режим ``Сопровождение'' для КРТ эквивалентен поддержанию постоянной
ориентации телескопа в пространстве и, как и два других режима, осуществляется с помощью
исполнительных органов системы управления ориентацией космического аппарата с контролем по
звездным датчикам, без использования реактивных двигателей при наблюдениях.
Наведение на источник и сканирование источника выполняется в прямоугольной системе координат
космического аппарата поворотами аппарата вокруг осей Y или Z. Пересчет в астрономическую
экваториальную систему координат осуществляется по телеметрическим данным от служебной системы
координатного обеспечения. В общем случае сканы источника в картинной плоскости представляют собой
практически прямые отрезки траекторий, отражающие сечение небесной сферы осью X
в системе экваториальных координат.

Методика проведения радиоастрономических антенных измерений унифицирована и идентична для каждой
длины волны. В каждом сеансе (обычно в течение около 2 ч) проводились измерения в одном из выбранных
режимов сканирования одновременно по всем задействованным диапазонам и поляризационным каналам (как
правило, по 2 поляризационных канала в 2 диапазонах). В зависимости от программы формирования кадров
и скорости опроса датчиков, задаваемых командами на телеметрическую систему, в серии из нескольких
последовательных кадров регистрируются все телеметрируемые параметры задействованной научной и
служебной аппаратуры, включая сигналы радиометрических аналоговых и цифровых выходов приемников,
коды бортовой шкалы времени и координатного обеспечения.


\section{Первые лепестки}

Статьи: \cite{Kardashev_2013_rus}



\section{Оценки точности восстановления орбиты КРТ по интерферометрическим наблюдениям}

Статьи: \cite{Lobanov_2015,Zakhvatkin_2018}


\section{Картографирование квазара 0716+714}

Статьи: \cite{Kardashev_2013_rus}

Для активной галактики 0716+714~--- одного из самых быстропеременных внегалактических объектов~---
успешное детектирование интерференционных лепестков в том же диапазоне было выполнено для многих
проекций баз, от примерно 1.5 до более 5 диаметров Земли. Международной рабочей группе по Ранней
научной программе проекта ``РадиоАстрон'' удалось по этим данным восстановить изображение объекта
(рис. ??) и измерить параметры видимого ядра. Ширина у основания струи в ядре объекта оказалась
равной примерно \SI{70}{\uas} или \SI{0.3}{\parsec}, а яркостная температура~--- \SI{2e12}{\kelvin}.
Заметим, что эти параметры измерены в момент минимума активности этого объекта.

\section{Выводы}

\begin{enumerate}
 \item
\end{enumerate}

