\chapter{Оформление различных элементов} \label{chapt1}


\section{Введение}
На изображениях внегалактических релятивистских струй, полученных с помощью Радиоинтерферометрии со
СверхДлинными Базами (РСДБ), ``ядром'' обычно называют наиболее компактную и яркую деталь у видимого
основания струи. Положение ядра определяется поглощением в излучающей плазме (синхротронное
самопоглощение) или поглощением в окружающем веществе
\cite{Blandford_Konigl_1979,Konigl_1981,Lobanov_1998}. На частоте наблюдения $\nu$ ядро находится в
области струи с оптической толщей $\tau(\nu) \approx 1$, что приводит к смещению абсолютного
положения ядра $r_\mathrm{core}$ как $\propto \nu^{-1/k_{\mathrm{r}}}$ \cite{Lobanov_1998}. В случае
синхротронного самопоглощения и при равнораспределении плотности энергии частиц и магнитного поля
$k_\mathrm{r} = 1$ \cite{Blandford_Konigl_1979}. Однако при наличии внешнего поглощения или
градиентов плотности и давления, $k_\mathrm{r}$ может отличаться от единицы \cite{Lobanov_1998}.

Эффект видимого сдвига ядра активных ядер галактик имеет непосредственные астрофизические и
астрометрические приложения в исследованиях компактных радиоисточников. Этот эффект может быть
использован для оценок
различных физических параметров компактных релятивистских струй. С другой стороны, сдвиг ядра с
частотой может влиять на измерения и оценки, основанные на многочастотных РСДБ наблюдениях: 1)
построение карт спектрального индекса \cite{Lobanov_1998,Kovalev_2008}; 2) измерения фарадеевского
вращения \cite{Hovatta_2012,Krav2016,Krav2017}; 3) астрометрические и геофизические измерения в
диапазонах 4 и 13~см \cite{Ma_1998,Petrov2009}; 4) сопоставление радио и оптической системы
координат \cite{PK_letter2017,KPP2017,PK2017}.

Для понимания и учета эффектов поглощения в подобных исследованиях необходимо систематическое
изучение смещения ядра, которое бы охватывало репрезентативную выборку компактных радиоисточников, в
частности, тех, которые используются в астрометрических приложениях. Для достижения этой цели мы
организовали пилотный эксперимент на Европейской РСДБ сети. Он преследовал следующие цели:
проведение измерений сдвига ядра у ультракомпактных квазаров методом относительной астрометрии в
рамках отобранных триплетов источников, сравнение результатов с традиционным методом
привязки к оптически тонким частям струи самого источника, получение опыта для использования в
организации более массовых астрометрических измерений данного эффекта в будущем, апробация участия
российской системы Квазар-КВО в наблюдениях Европейской РСДБ сети и оценка выигрыша от такого
участия.
