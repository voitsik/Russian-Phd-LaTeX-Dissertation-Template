%%% Основные сведения %%%
\newcommand{\thesisAuthorLastName}{Войцик}
\newcommand{\thesisAuthorOtherNames}{Пётр Андреевич}
\newcommand{\thesisAuthorInitials}{П.\,А.}
\newcommand{\thesisAuthor}             % Диссертация, ФИО автора
{%
    \texorpdfstring{% \texorpdfstring takes two arguments and uses the first for (La)TeX and the second for pdf
        \thesisAuthorLastName~\thesisAuthorOtherNames% так будет отображаться на титульном листе или в тексте, где будет использоваться переменная
    }{%
        \thesisAuthorLastName, \thesisAuthorOtherNames% эта запись для свойств pdf-файла. В таком виде, если pdf будет обработан программами для сбора библиографических сведений, будет правильно представлена фамилия.
    }
}
\newcommand{\thesisAuthorShort}        % Диссертация, ФИО автора инициалами
{\thesisAuthorInitials~\thesisAuthorLastName}
%\newcommand{\thesisUdk}                % Диссертация, УДК
%{\todo{xxx.xxx}}
\newcommand{\thesisTitle}              % Диссертация, название
{Исследование центральных областей ядер активных галактик}
\newcommand{\thesisSpecialtyNumber}    % Диссертация, специальность, номер
{01.03.02}
\newcommand{\thesisSpecialtyTitle}     % Диссертация, специальность, название
{Астрофизика и звёздная астрономия}
\newcommand{\thesisDegree}             % Диссертация, ученая степень
{кандидата физико-математических наук}
\newcommand{\thesisDegreeShort}        % Диссертация, ученая степень, краткая запись
{канд. физ.-мат. наук}
\newcommand{\thesisCity}               % Диссертация, город написания диссертации
{Москва}
\newcommand{\thesisYear}               % Диссертация, год написания диссертации
{2018}
\newcommand{\thesisOrganization}       % Диссертация, организация
{РОССИЙСКАЯ АКАДЕМИЯ НАУК\\
ФИЗИЧЕСКИЙ ИНСТИТУТ ИМЕНИ П. Н. ЛЕБЕДЕВА\\
АСТРОКОСМИЧЕСКИЙ ЦЕНТР
}
\newcommand{\thesisOrganizationShort}  % Диссертация, краткое название организации для доклада
{АКЦ ФИАН}

\newcommand{\thesisInOrganization}     % Диссертация, организация в предложном падеже: Работа выполнена в ...
{\todo{учреждении, в~котором выполнялась данная диссертационная работа}}

\newcommand{\supervisorFio}            % Научный руководитель, ФИО
{Ковалев Юрий Юрьевич}
\newcommand{\supervisorRegalia}        % Научный руководитель, регалии
{доктор физико-математических наук,\\
профессор РАН,\\
член-корреспондент РАН}
\newcommand{\supervisorFioShort}       % Научный руководитель, ФИО
{Ю.\,Ю.~Ковалев}
\newcommand{\supervisorRegaliaShort}   % Научный руководитель, регалии
{док. физ.-мат. наук, член-корр. РАН}


\newcommand{\opponentOneFio}           % Оппонент 1, ФИО
{\todo{Фамилия Имя Отчество}}
\newcommand{\opponentOneRegalia}       % Оппонент 1, регалии
{\todo{доктор физико-математических наук, профессор}}
\newcommand{\opponentOneJobPlace}      % Оппонент 1, место работы
{\todo{Не очень длинное название для места работы}}
\newcommand{\opponentOneJobPost}       % Оппонент 1, должность
{\todo{старший научный сотрудник}}

\newcommand{\opponentTwoFio}           % Оппонент 2, ФИО
{\todo{Фамилия Имя Отчество}}
\newcommand{\opponentTwoRegalia}       % Оппонент 2, регалии
{\todo{кандидат физико-математических наук}}
\newcommand{\opponentTwoJobPlace}      % Оппонент 2, место работы
{\todo{Основное место работы c длинным длинным длинным длинным названием}}
\newcommand{\opponentTwoJobPost}       % Оппонент 2, должность
{\todo{старший научный сотрудник}}

\newcommand{\leadingOrganizationTitle} % Ведущая организация, дополнительные строки
{Московский государственный университет им. М. В. Ломоносова (МГУ, Государственный астрономический институт им. П. К. Штернберга), г. Москва
}

\newcommand{\defenseDate}              % Защита, дата
{\todo{DD mmmmmmmm YYYY~г.~в~XX часов}}
\newcommand{\defenseCouncilNumber}     % Защита, номер диссертационного совета
{\todo{Д\,123.456.78}}
\newcommand{\defenseCouncilTitle}      % Защита, учреждение диссертационного совета
{\todo{Название учреждения}}
\newcommand{\defenseCouncilAddress}    % Защита, адрес учреждение диссертационного совета
{\todo{Адрес}}
\newcommand{\defenseCouncilPhone}      % Телефон для справок
{\todo{+7~(0000)~00-00-00}}

\newcommand{\defenseSecretaryFio}      % Секретарь диссертационного совета, ФИО
{Ю.А.~Ковалев}
\newcommand{\defenseSecretaryRegalia}  % Секретарь диссертационного совета, регалии
{д-р~физ.-мат. наук}            % Для сокращений есть ГОСТы, например: ГОСТ Р 7.0.12-2011 + http://base.garant.ru/179724/#block_30000

\newcommand{\synopsisLibrary}          % Автореферат, название библиотеки
{\todo{Название библиотеки}}
\newcommand{\synopsisDate}             % Автореферат, дата рассылки
{\todo{DD mmmmmmmm YYYY года}}

% To avoid conflict with beamer class use \providecommand
\providecommand{\keywords}%            % Ключевые слова для метаданных PDF диссертации и автореферата
{}
