
{\actuality}

Активные ядра галактик ...

% \ifsynopsis
% Этот абзац появляется только в~автореферате.
% Для формирования блоков, которые будут обрабатываться только в~автореферате,
% заведена проверка условия \verb!\!\verb!ifsynopsis!.
% Значение условия задаётся в~основном файле документа (\verb!synopsis.tex! для
% автореферата).
% \else
% Этот абзац появляется только в~диссертации.
% Через проверку условия \verb!\!\verb!ifsynopsis!, задаваемого в~основном файле
% документа (\verb!dissertation.tex! для диссертации), можно сделать новую
% команду, обеспечивающую появление цитаты в~диссертации, но~не~в~автореферате.
% \fi

% {\progress} 
% Этот раздел должен быть отдельным структурным элементом по
% ГОСТ, но он, как правило, включается в описание актуальности
% темы. Нужен он отдельным структурынм элемементом или нет ---
% смотрите другие диссертации вашего совета, скорее всего не нужен.

{\aim}

Целью данной работы является \ldots

{\novelty}
% \begin{enumerate}
%   \item Впервые \ldots
%   \item Впервые \ldots
%   \item Было выполнено оригинальное исследование \ldots
% \end{enumerate}

% {\influence} \ldots

% {\methods} \ldots

{\defpositions}
\begin{enumerate}
  \item
  \item
  \item
\end{enumerate}


{\reliability}

{\probation}


%%% Реализация пакетом biblatex через движок biber
    \begin{refsection}[bl-author]
        % Это refsection=1.
        % Процитированные здесь работы:
        %  * подсчитываются, для автоматического составления фразы "Основные результаты ..."
        %  * попадают в авторскую библиографию, при usefootcite==0 и стиле `\insertbiblioauthor` или `\insertbiblioauthorgrouped`
        %  * нумеруются там в зависимости от порядка команд `\printbibliography` в этом разделе.
        %  * при использовании `\insertbiblioauthorgrouped`, порядок команд `\printbibliography` в нём должен быть тем же (см. biblio/biblatex.tex)
        %
        % Невидимый библиографический список для подсчёта количества публикаций:
        \printbibliography[heading=nobibheading, section=1, env=countauthorvak,          keyword=biblioauthorvak]%
%         \printbibliography[heading=nobibheading, section=1, env=countauthorwos,          keyword=biblioauthorwos]%
%         \printbibliography[heading=nobibheading, section=1, env=countauthorscopus,       keyword=biblioauthorscopus]%
        \printbibliography[heading=nobibheading, section=1, env=countauthorconf,         keyword=biblioauthorconf]%
        \printbibliography[heading=nobibheading, section=1, env=countauthorother,        keyword=biblioauthorother]%
        \printbibliography[heading=nobibheading, section=1, env=countauthor,             keyword=biblioauthor]%
%         \printbibliography[heading=nobibheading, section=1, env=countauthorvakscopuswos, filter=vakscopuswos]%
%         \printbibliography[heading=nobibheading, section=1, env=countauthorscopuswos,    filter=scopuswos]%
        %
        \nocite{*}%
        %
%         {\publications} Основные результаты по теме диссертации изложены в~\arabic{citeauthor}~печатных изданиях,
%         \arabic{citeauthorvak} из которых изданы в журналах, рекомендованных ВАК\sloppy%
%         \ifnum \value{citeauthorscopuswos}>0%
%             , \arabic{citeauthorscopuswos} "--- в~периодических научных журналах, индексируемых Web of~Science и Scopus\sloppy%
%         \fi%
%         \ifnum \value{citeauthorconf}>0%
%             , \arabic{citeauthorconf} "--- в~тезисах докладов.
%         \else%
%             .
%         \fi
    \end{refsection}%

\section*{\bibtitleauthor}
Основные результаты по теме диссертации изложены в~\arabic{citeauthor}~научных статьях
\cite{Voitsik_2018, Kardashev_2013_rus, Kovalev_2014_rus, Lobanov_2015, Kutkin_2018}, в
рецензируемых журналах и изданиях, рекомендованных ВАК.
% \insertbiblioauthorcited

\insertbiblioauthor

{\contribution}

Глава 1: 9

9: В дисер идет всё, кроме таблицы 3.
Во вкладе: свое (обработка, свой анализ) - определяющий вклад, написание статьи - основной вклад,
остальное - равный вклад  с другими соавторами.

Глава 2: 2, 3, 4, 11.

2: вклад: участие в получении первых лепестков, корреляции и получении карты 0716, чувствительность,
эфф площадь, SEFD, для КРТ.

3: вклад: участие в измерении чувствительность, эфф площадь, SEFD, для КРТ.
4: равный вклад в анализ лепестков SVLBI, основной вклад по анализу остаточной скорости - точности
орбиты.
11: основной вклад: посткорреляционная обработка данных РА обзора AGN, точное определение остаточной
скорости КРТ, равный вклад про годовые циклы, точность и сравнение орбит.

Глава 3: 1, 5, 6, 7, 8, 10

1: в диссертацию идет все, кроме ускорения.
Вклад: равный с другими соавторами вклад в подгонку моделей в uv данные, измерение кинематики джетов
и анализ.

5, 6, 7, 8, 10: основной вклад в посткорреляционную обработку данных РА наблюдения активных ядер
галактик, в оценку яркостных температур, равное участие в обсуждении результатов, анализе, написании
статей.



% \section*{Объем и структура диссертации}
%
% Диссертация состоит из введения, трёх глав, заключения и списка литературы.
% Полный объём диссертации составляет
% \formbytotal{TotPages}{страниц}{у}{ы}{}, включая
% \formbytotal{totalcount@figure}{рисун}{ок}{ка}{ков} и
% \formbytotal{totalcount@table}{таблиц}{у}{ы}{}.   Список литературы содержит
% \formbytotal{citenum}{наименован}{ие}{ия}{ий}.
%
